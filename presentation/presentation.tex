\documentclass{beamer}
\usetheme{Metropolis}
\metroset{numbering=none}
\metroset{sectionpage=none}
\setbeamertemplate{section in toc}[sections numbered]
\author{Saugat The Great}
\title{Anomaly Detection with Isolation Forest and Segmentation with K-means Clustering}
\subtitle{A Brief Introduction to Unsupervised Learning in Student Data Analysis}
\addtobeamertemplate{section in toc}{\vspace{-0.5ex}}{}
\begin{document}
	
	% -----------------------------------------------------------
	% Title Page and Outline
	% -----------------------------------------------------------
	\begin{frame}
		\titlepage
	\end{frame}
	
	\section{Outline} % This section contains the ToC itself
	\begin{frame}{Presentation Outline}
		\small
		\vspace{0.3cm}
		\tableofcontents
	\end{frame}
	
	% -----------------------------------------------------------
	% Abstract and Introduction
	% FIX: Make sections/subsections consistent for the ToC
	% -----------------------------------------------------------
	\section{Abstract and Introduction}
	\subsection{Project Overview} % Added for consistency
	\begin{frame}{Abstract}
		The presentation explores the application of unsupervised machine learning techniques to analyze student interaction data.
		\par
		\vspace{0.5em}
		\begin{itemize}
			\item The workflow focused on robust data preprocessing (handling, cleaning, and feature extraction) to prepare the dataset.
			\item Anomaly Detection was performed using the highly effective Isolation Forest algorithm to identify irregular student behaviors.
			\item Segmentation was achieved with K-means Clustering to group students into $\mathbf{K=4}$ distinct behavioral clusters for targeted intervention and analysis.
		\end{itemize}
	\end{frame}
	
	\begin{frame}{Introduction}
		\small
		\frametitle{Project Focus and Methodology} % Use frametitle for a clean look
		Project Focus: This project details the comprehensive process of data ingestion, cleaning, transformation, and analysis. The core objective is the implementation of Anomaly Detection and data segmentation.
		\par
		\vspace{0.5em}
		Methodology: The workflow emphasizes robust data preprocessing (handling, cleaning, and manipulation) to prepare the dataset before applying the machine learning models.
	\end{frame}
	
	\begin{frame}{Introduction}
		\small
		\frametitle{Key Technologies and Libraries Used (Part 1/2)} % Use frametitle for a clean look
		Key Technologies and Libraries:
		\begin{enumerate}
			\item NumPy: Essential for high-performance numerical computation and serving as the foundational internal data structure for array manipulation.
			\item Pandas: Used for efficient data loading, handling, and analysis via DataFrames.
			\item Matplotlib: Primary library for generating static and interactive visualizations.
		\end{enumerate}
	\end{frame}
	
	\begin{frame}{Introduction}
		\small
		\frametitle{Key Technologies and Libraries Used (Part 2/2)}
		\par
		Key Technologies and Libraries:
		\begin{enumerate}
			\setcounter{enumi}{3} % Starts the list at number 4
			\item Seaborn: Statistical visualization library; provides a high-level interface for attractive statistical graphics.
			\item Scikit-learn (sklearn): Comprehensive library for implementing the machine learning algorithms, including Isolation Forest and K-means.
		\end{enumerate}
	\end{frame}
	
	\metroset{sectionpage=progressbar}
	
	% -----------------------------------------------------------
	% Person 1: Data Handling and Cleaning
	% -----------------------------------------------------------
	\section{Data Handling and Cleaning}
	\subsection{Data Overview and Filtering} % Consolidate subsections for a cleaner ToC
	\begin{frame}{Presenter 1: Data Overview and Cleaning}
		\frametitle{Data Overview and Initial Preparation}
		Source of the Data
		\begin{itemize}
			\item The data is collected from \href{https://archive.ics.uci.edu/dataset/1031/dataset+for+assessing+mathematics+learning+in+higher+education}{UCI Machine Learning Repository}.
			\item Initial total number of individual interactions: $\mathbf{9546}$.
			\item The dataset contains $\mathbf{8}$ columns, capturing student country, question metadata (ID, level, topic, subtopic, keywords), and interaction outcome (\texttt{is\_correct}).
		\end{itemize}
		
		Data Information:
		\begin{itemize}
			\item Handling Missing Values: A full check revealed no missing values, thus no imputation techniques were necessary.
			\item Column Renaming: Columns were renamed for enhanced readability and ease of programming, e.g., $\text{'Student ID'} \rightarrow \text{'student\_id'}$.
		\end{itemize}
	\end{frame}
	
	\begin{frame}{Handling Duplicate and Low-Activity Data (P1)}
		\frametitle{Duplicate Data Filtering}
		Handling Duplicate Values
		\begin{itemize}
			\item $\mathbf{2083}$ rows were found to be identical instances of data.
			\item Rationale: Given the nature of a single student interaction being logged, identical rows likely represent data collection errors rather than simultaneous events. These duplicates were removed.
			\item $\text{Final Interaction Count} = 9546 - 2083 = \mathbf{7463}$ interactions remaining.
		\end{itemize}
	\end{frame}
	
	\begin{frame}{Handling Duplicate and Low-Activity Data (P1)}
		\frametitle{Low-Activity Student Filtering}
		Filtering Low-Activity Students
		\begin{itemize}
			\item Students with fewer than a minimum number of interactions ($\mathbf{MIN\_ATTEMPTS = 5}$) were removed from the analysis.
			\item Rationale: Students with very few attempts do not provide enough data to form reliable behavioral features, and their inclusion could skew the clustering and anomaly detection processes.
		\end{itemize}
		This ensures the final dataset for feature engineering represents only students with meaningful activity.
	\end{frame}
	
	% -----------------------------------------------------------
	% Person 2: Feature Extraction
	% -----------------------------------------------------------
	\section{Feature Engineering}
	\subsection{Feature Creation and Standardization} % Consolidate subsections for a cleaner ToC
	\begin{frame}{Presenter 2: Feature Engineering and Normalization}
		\frametitle{Feature Engineering: Step 1 - Aggregation}
		The raw interaction data (long format) was aggregated to create a single feature vector (wide format) for each unique student ($\sim 372$ students), forming the basis for analysis.
		\par
		\vspace{0.5em}
		Key Features Extracted ($\mathbf{FEATURE\_COLS}$):
		\begin{itemize}
			\item Overall Correct Ratio ($\mathbf{CR}$): Measures general student performance.
			\begin{equation*}
				\mathbf{CR} = \frac{\sum \text{Correct Attempts}}{\sum \text{Total Attempts}}
			\end{equation*}
			\item Total Attempts: Student's overall volume of engagement.
		\end{itemize}
		This process transforms the focus from individual attempts to student-level behavior.
	\end{frame}
	
	\begin{frame}{Presenter 2: Feature Engineering and Normalization}
		\frametitle{Feature Engineering: Step 2 - Diversity Metrics}
		\vspace{-1em}
		Key Features Extracted (Continued):
		\begin{itemize}
			\item Unique Diversity Metrics: Count of unique Question IDs, Topics, Subtopics, and Keywords attempted. Measures the breadth of student engagement.
		\end{itemize}
		\vfill
		Rationale for Feature Engineering:
		\begin{itemize}
			\item The clustering and anomaly detection algorithms require a single, rich profile for each student, not a long list of individual attempts.
			\item These aggregated features (Correct Ratio, Attempts, Diversity) capture the fundamental dimensions of student behavior: performance, volume, and scope.
		\end{itemize}
	\end{frame}
	\begin{frame}{Data Standardization (P2)}
		\frametitle{Standardization for Model Readiness}
		Machine learning algorithms like K-means and Isolation Forest are highly sensitive to the scale of features. Features with a larger magnitude (e.g., Total Attempts) would unfairly dominate the distance calculations.
		\par
		Technique: Standard Scaling (Z-score Normalization)
		\begin{itemize}
			\item The features were standardized to have a mean of $\mathbf{0}$ and a standard deviation of $\mathbf{1}$.
			\item For a feature value $x$, the standardized value $z$ is:
			\begin{equation*}
				z = \frac{x - \mu}{\sigma}
			\end{equation*}
			Where $\mu$ is the mean and $\sigma$ is the standard deviation of the feature across all students.
		\end{itemize}
		This ensures all features contribute equally to the distance metrics used in the subsequent algorithms.
	\end{frame}
	
	% -----------------------------------------------------------
	% Person 3: Isolation Forest
	% -----------------------------------------------------------
	\section{Isolation Forest for Anomaly Detection}
	\subsection{Theory, Math, and Implementation} % Consolidate subsections for a cleaner ToC
	\begin{frame}{Presenter 3: Isolation Forest - The Theory}
		\frametitle{Isolation Forest: Detecting Anomalous Student Behavior}
		Goal: Identify students whose feature vectors ($\mathbf{x} \in \mathbb{R}^d$) are significantly different from the majority.
		\par
		Concept: Unlike distance-based methods that try to define a 'normal' region, Isolation Forest (iForest) works by explicitly isolating anomalies.
		\par
		Mechanism: iForest is an ensemble of Isolation Trees (iTrees).
		\begin{itemize}
			\item It randomly selects a feature and a split value, recursively partitioning the data.
			\item Anomalies are few and far from the bulk of the data, meaning they are easier to isolate and require fewer random partitions (splits) to reach a terminal node.
			\item Normal points are numerous and require more splits to be isolated.
		\end{itemize}
	\end{frame}
	
	\begin{frame}{Mathematical Basis of Isolation Forest (P3)}
		\frametitle{iForest: Path Length and Anomaly Score Formula}
		The measure of abnormality is based on the Path Length $\mathbf{h(x)}$ from the root of an iTree to the terminal node containing the instance $\mathbf{x}$.
		\par
		\vspace{0.5em}
		1. Average Path Length ($\mathbf{E[h(x)]}$):
		\begin{itemize}
			\item The path length is averaged across all iTrees in the forest.
			\item A shorter average path length indicates an anomaly (easier to isolate).
		\end{itemize}
		\vfill
		2. Anomaly Score Formula ($\mathbf{s(x)}$):
		The raw path length is converted into an anomaly score $s(x) \in [0, 1]$:
		\begin{equation*}
			s(x) = 2^{-\frac{E[h(x)]}{c(n)}}
		\end{equation*}
		Where $\mathbf{c(n)}$ is a normalization factor related to the sample size $\mathbf{n}$.
	\end{frame}
	
	\begin{frame}{Mathematical Basis of Isolation Forest (P3)}
		\frametitle{Interpreting the Anomaly Score}
		Interpretation of Score ($\mathbf{s(x)}$):
		\begin{itemize}
			\item $s(x) \approx 1$: Indicates an anomaly (very short average path length).
			\item $s(x) < 0.5$: Indicates a normal instance (longer average path length).
			\item $s(x) \approx 0.5$: Indicates no distinct anomaly.
		\end{itemize}
		\vfill
		Detection Strategy:
		Instances with scores above a chosen threshold are flagged as anomalies.
		\par
		\vspace{0.5em}
		Implementation Detail (Contamination):
		The model was run with a predefined $\mathbf{contamination}$ rate of $\mathbf{0.03}$ ($\mathbf{3\%}$). This parameter implicitly sets the threshold to classify the top 3\% highest-scoring instances as anomalies.
	\end{frame}
	
	% -----------------------------------------------------------
	% Person 4: K-means Clustering
	% -----------------------------------------------------------
	\section{K-means Clustering}
	\subsection{Algorithm and Evaluation} % Consolidate subsections for a cleaner ToC
	\begin{frame}{Presenter 4: K-means Clustering for Segmentation}
		\frametitle{K-means Clustering: Segmenting Student Behavior}
		Goal: Partition the student feature vectors into $\mathbf{K}$ groups (clusters), where each student belongs to the cluster with the nearest mean (centroid).
		\par
		The K-means Algorithm Steps:
		\begin{enumerate}
			\item Initialization: Select $\mathbf{K}$ initial cluster centroids (randomly or using $\mathbf{k++}$ initialization).
			\item Assignment (E-Step): Assign each student vector $\mathbf{x}$ to the cluster $\mathbf{S_j}$ whose centroid $\mathbf{\mu_j}$ is the closest (using $\mathbf{L_2}$ Euclidean distance).
			\item Update (M-Step): Recalculate the centroid $\mathbf{\mu_j}$ for each cluster $S_j$ as the mean of all points assigned to it.
			\item Convergence: Repeat steps 2 and 3 until the centroids no longer change or a maximum number of iterations is reached.
		\end{enumerate}
	\end{frame}
	
	\begin{frame}{K-means Objective Function (P4)}
		\frametitle{Mathematical Objective: Within-Cluster Sum of Squares (WCSS)}
		K-means minimizes the Within-Cluster Sum of Squares (WCSS), also known as Inertia. This ensures clusters are compact.
		\vfill
		\begin{equation*}
			J = \sum_{j=1}^{K} \sum_{\mathbf{x} \in S_j} \|\mathbf{x} - \mathbf{\mu_j}\|^2
		\end{equation*}
		\vfill
		Where:
		\begin{itemize}
			\item $J$ is the Inertia (WCSS).
			\item $K$ is the number of clusters.
			\item $\mathbf{x}$ is a data point (student feature vector).
			\item $\mathbf{\mu_j}$ is the centroid of cluster $S_j$.
			\item $\|\dots\|^2$ is the squared Euclidean distance.
		\end{itemize}
	\end{frame}
	
	\begin{frame}{Cluster Selection and Evaluation (P4)}
		\frametitle{Selecting $\mathbf{K}$ and Evaluating Cluster Quality}
		Cluster Number Selection ($\mathbf{K}$):
		\begin{itemize}
			\item The Elbow Method (plotting WCSS vs. $\mathbf{K}$) is used to find the "bend" point where diminishing returns are observed.
			\item
			\item For this analysis, $\mathbf{K=4}$ was chosen based on combining the Elbow Method result with the highest Silhouette Score for a robust and interpretable set of segments.
		\end{itemize}
		\vfill
		Evaluation Metric: Silhouette Score ($\mathbf{s}$)
		The score measures how similar an object is to its own cluster compared to other clusters.
		\begin{equation*}
			s = \frac{b - a}{\max(a, b)}
		\end{equation*}
		Where $\mathbf{a}$ is the mean intra-cluster distance and $\mathbf{b}$ is the mean nearest-cluster distance. Scores closer to $\mathbf{+1}$ indicate dense, well-separated clusters.
	\end{frame}
	
	\begin{frame}{Cluster Profile Interpretation (P4)}
		\frametitle{Interpreting the $\mathbf{K=4}$ Student Clusters}
		The final step is to interpret the meaning of the clusters by analyzing their Centroids (mean feature values) for the $\mathbf{K=4}$ segments.
		\par
		
		\par
		Example Cluster Characteristics (based on centroid analysis):
		\begin{itemize}
			\item Cluster 0: The Engaged High Achievers (High $\mathbf{CR}$, very high Diversity)
			\item Cluster 1: The Struggling but Persistent (Low $\mathbf{CR}$, high Total Attempts)
			\item Cluster 2: The Focused Low Engagers (Average $\mathbf{CR}$, low Diversity and Total Attempts)
			\item Cluster 3: The Efficient High Achievers (Very high $\mathbf{CR}$, moderate Attempts)
		\end{itemize}
	\end{frame}
	
	% -----------------------------------------------------------
	% Conclusion
	% -----------------------------------------------------------
	\section{Conclusion}
	\subsection{Summary and Future Work} % Added for consistency
	\begin{frame}{Conclusion and Future Work}
		\frametitle{Summary of Findings}
		\begin{itemize}
			\item The data was successfully preprocessed, resulting in a clean dataset of $\mathbf{7463}$ interactions from robust students.
			\item Isolation Forest effectively flagged a small fraction ($\mathbf{3\%}$) of students as anomalous, suggesting they exhibit highly irregular behavior compared to the norm.
			\item K-means Clustering successfully segmented the students into $\mathbf{4}$ distinct behavioral groups, which can be used to tailor educational strategies (e.g., targeted assistance for struggling clusters).
		\end{itemize}
		\par
		Future Work:
		\begin{itemize}
			\item Explore DBSCAN or OPTICS clustering for natural density-based groups.
			\item Incorporate temporal features (e.g., time between attempts) into the feature set.
		\end{itemize}
	\end{frame}
	
\end{document}